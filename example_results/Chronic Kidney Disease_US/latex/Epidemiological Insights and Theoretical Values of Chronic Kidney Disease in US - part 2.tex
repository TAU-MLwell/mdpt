\documentclass{article}

% Packages
\usepackage[utf8]{inputenc}
\usepackage{hyperref}

% Title and Author
\title{Chronic Kidney Disease: Epidemiology, Medical Context, and Policy Changes}
\author{}
\date{\today}

\begin{document}

\maketitle

\begin{abstract}
Chronic Kidney Disease (CKD) is a significant public health issue both in the United States and globally. This document provides an overview of CKD, including its epidemiology, medical context, management guidelines, and policy changes.
\end{abstract}

\section{Introduction}
Chronic Kidney Disease (CKD) is characterized by a gradual loss of kidney function over time, which can lead to end-stage renal disease (ESRD) requiring dialysis or kidney transplantation.

\section{Epidemiology}
\subsection{Prevalence}
CKD affects approximately 15\% of the adult population in the United States, according to the Centers for Disease Control and Prevention (CDC). Worldwide, the prevalence varies but is estimated to affect about 10\% of the global population.

\subsection{Risk Factors}
Major risk factors include diabetes, hypertension, cardiovascular disease, obesity, and a family history of kidney disease. Age, ethnicity, and socioeconomic factors also play a role, with higher prevalence observed in African American, Hispanic, and Native American populations.

\section{Medical Context}
\subsection{Diagnosis}
CKD is diagnosed based on the presence of kidney damage or decreased kidney function for three months or more, using markers such as albuminuria and estimated glomerular filtration rate (eGFR). The National Kidney Foundation (NKF) provides guidelines for staging CKD from stage 1 (mild) to stage 5 (severe).

\subsection{Symptoms}
Early stages may be asymptomatic, but as the disease progresses, symptoms can include fatigue, swelling, changes in urination, and high blood pressure.

\section{Guidelines in the US}
\subsection{Management}
The Kidney Disease: Improving Global Outcomes (KDIGO) and the NKF provide guidelines for the management of CKD, emphasizing blood pressure control, glycemic management in diabetes, lifestyle modifications, and regular monitoring of kidney function.

\subsection{Treatment}
Treatment strategies include the use of medications such as ACE inhibitors or ARBs to slow progression, dietary modifications, and management of complications like anemia and bone mineral disorders.

\section{Policy Changes}
\subsection{Global Policy Changes}
\begin{itemize}
    \item \textbf{2002 - Kidney Disease: Improving Global Outcomes (KDIGO):} Established to develop and implement global clinical practice guidelines in nephrology.
    \item \textbf{2004 - World Health Organization (WHO) Resolution on Chronic Diseases:} Recognized CKD as part of the global burden of chronic diseases.
    \item \textbf{2012 - KDIGO CKD Guidelines Update:} Provided updated guidelines on the evaluation and management of CKD.
\end{itemize}

\subsection{Local Policy Changes in the United States}
\begin{itemize}
    \item \textbf{2000 - National Kidney Foundation (NKF) Kidney Disease Outcomes Quality Initiative (KDOQI):} Released the first clinical practice guidelines for CKD.
    \item \textbf{2006 - Medicare Coverage for CKD Education:} CMS began covering kidney disease education services for patients with CKD stages 4 and 5.
    \item \textbf{2017 - Advancing American Kidney Health Initiative:} Launched by HHS to improve the prevention, diagnosis, and treatment of kidney diseases.
\end{itemize}

\section*{References}
\begin{itemize}
    \item Centers for Disease Control and Prevention (CDC) - Chronic Kidney Disease Surveillance System: \url{https://www.cdc.gov/kidneydisease/publications-resources/ckd-national-facts.html}
    \item National Kidney Foundation (NKF) - Kidney Disease Outcomes Quality Initiative (KDOQI) Guidelines: \url{https://www.kidney.org/professionals/guidelines}
    \item Kidney Disease: Improving Global Outcomes (KDIGO) - Clinical Practice Guidelines: \url{https://kdigo.org/guidelines/}
    \item World Health Organization (WHO) - Global Strategy on Diet, Physical Activity and Health: \url{https://www.who.int/dietphysicalactivity/strategy/eb11344/strategy_english_web.pdf}
    \item U.S. Centers for Medicare & Medicaid Services (CMS) - Medicare Coverage of Kidney Disease Education: \url{https://www.cms.gov/Outreach-and-Education/Medicare-Learning-Network-MLN/MLNProducts/preventive-services/medicare-wellness-visits.html}
    \item U.S. Department of Health and Human Services (HHS) - Advancing American Kidney Health Initiative: \url{https://aspe.hhs.gov/reports/advancing-american-kidney-health}
\end{itemize}

\end{document}