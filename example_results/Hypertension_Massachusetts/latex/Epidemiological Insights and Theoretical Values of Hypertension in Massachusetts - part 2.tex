\documentclass{article}
\usepackage{geometry}
\usepackage{hyperref}
\usepackage{titlesec}

\geometry{a4paper, margin=1in}

\title{Hypertension: An Overview and Policy Changes}
\author{}
\date{}

\begin{document}

\maketitle

\section{Hypertension Overview}
Hypertension, commonly known as high blood pressure, is a condition where the force of the blood against the artery walls is consistently too high. It is a major risk factor for cardiovascular diseases, including stroke, heart attack, and heart failure.

\section{Epidemiology}
Globally, hypertension affects approximately 1.28 billion adults aged 30-79 years, with a significant portion unaware of their condition. In the United States, including Massachusetts, hypertension is prevalent in nearly half of the adult population. The prevalence increases with age and is influenced by lifestyle factors such as diet, physical activity, and stress levels.

\section{Medical Context in Massachusetts}
In Massachusetts, hypertension is a significant public health concern. The Massachusetts Department of Public Health (MDPH) actively monitors and addresses hypertension through various initiatives aimed at prevention, early detection, and management. The state follows guidelines set by national bodies such as the American Heart Association (AHA) and the American College of Cardiology (ACC).

\section{Guidelines}
The AHA/ACC guidelines recommend regular blood pressure screenings for adults, lifestyle modifications (such as a balanced diet, regular exercise, and smoking cessation), and pharmacological treatment when necessary. The target blood pressure is generally less than 130/80 mmHg for most adults.

\section{Management Strategies}
\subsection{Lifestyle Modifications}
Emphasizing the DASH diet (Dietary Approaches to Stop Hypertension), reducing sodium intake, increasing physical activity, and managing stress.

\subsection{Pharmacological Treatment}
When lifestyle changes are insufficient, medications such as ACE inhibitors, angiotensin II receptor blockers (ARBs), calcium channel blockers, and diuretics may be prescribed.

\section{Global and Local Policy Changes in Hypertension Diagnosis and Treatment}

\subsection{Global Policy Changes}
\begin{itemize}
    \item \textbf{2003:} The World Health Organization (WHO) and the International Society of Hypertension (ISH) released guidelines emphasizing the importance of lifestyle modifications and the use of combination drug therapy for better blood pressure control.
    \item \textbf{2013:} The Eighth Joint National Committee (JNC 8) provided updated guidelines, recommending less aggressive treatment targets for older adults and emphasizing individualized treatment plans.
    \item \textbf{2017:} The American College of Cardiology (ACC) and the American Heart Association (AHA) released new guidelines lowering the threshold for hypertension diagnosis from 140/90 mmHg to 130/80 mmHg, significantly increasing the number of individuals classified as having hypertension.
\end{itemize}

\subsection{Local Policy Changes in Massachusetts}
\begin{itemize}
    \item \textbf{2000s:} Massachusetts Department of Public Health (MDPH) began implementing community-based programs to increase awareness and control of hypertension, focusing on lifestyle interventions.
    \item \textbf{2010:} Massachusetts adopted the Million Hearts initiative, a national effort to prevent 1 million heart attacks and strokes by improving hypertension control through public health and clinical strategies.
    \item \textbf{2017:} Following the ACC/AHA guideline update, Massachusetts healthcare providers began adopting the new lower threshold for hypertension diagnosis, leading to increased screening and management efforts.
\end{itemize}

\section{References}
\begin{enumerate}
    \item American Heart Association (AHA) - Hypertension Guidelines
    \item American College of Cardiology (ACC) - Hypertension Management
    \item Massachusetts Department of Public Health - Hypertension Initiatives
    \item World Health Organization (WHO) - Hypertension Guidelines
    \item Eighth Joint National Committee (JNC 8) - Hypertension Management Guidelines
    \item American College of Cardiology (ACC) and American Heart Association (AHA) - 2017 Hypertension Guidelines
    \item Massachusetts Department of Public Health - Hypertension Initiatives and Million Hearts Program
\end{enumerate}

For further information, consult the Massachusetts Department of Public Health website and the latest AHA/ACC guidelines on hypertension management.

\end{document}